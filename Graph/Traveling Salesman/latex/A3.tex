\documentclass[]{article}
\usepackage{amsmath}
\usepackage{pgfplotstable}
\usepackage{pgfplots}
\usepackage{listings}
\pagestyle{empty}

%opening
\title{ECE345 A3}
\author{Jason Wang, Kevin Grafstrum, Oliver S.}

\begin{document}

\maketitle
\pagebreak
\section*{Question 8}
\subsection{Implementation Details}
	\textbf{How are you representing the graph, using an adjacency matrix or an adjacency list, and why?}\\
	I am representing the graph using an adjacency list. The social network is a sparse graph and using a adjacency matrix will require memory complexity of $m^2$\\
	\textbf{Which shortest path algorithm did you use and why?}\\
	I used Johnson's algorithm to find all the shortest paths, but I modified it so that it only searches the paths up to the required time limit. Johnson's algorithm is basically iterating Djikstra's algorithm through all the nodes. It has a runtime complexity of $O(|V|^{2}\log |V|+|V||E|)$\\
	
	\textbf{Graph of load factor vs Collision Count (Experiments)}\\
	
	\vspace{1cm}
	
	\begin{tikzpicture}[scale = 1.2]
	\begin{axis}[
	xlabel=Density,
	ylabel=Time]
	
	\addplot table [x=lf, y=$Run_time$, only marks]{../test/data.dat};
	%\addplot table[y={create col/linear regression={y=$Run_time$}}]{../test/data.dat};
	\addlegendentry{Time}
	\addlegendentry{Regression}
	
	\end{axis}
	\end{tikzpicture}
	
	\textbf{Discuss the plot above. What do you observe and why?}\\
	From the data, the computation time increases quadratically with the density of the graph. This is because Djistra's algorithm runs in $O(|E|+|V|\log |V|)$ but the influence distance will also increase linearly number of edges as more nodes become influenced. This gives an amortized complexity of $O(|E|^2+|E|(|V|\log |V|))$ where the $E$ term dominates and gives a quadratic.
	
\end{document}
